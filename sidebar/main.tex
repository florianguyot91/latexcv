%-----------------------------------------------------------------------------------------------------------------------------------------------%
%	The MIT License (MIT)
%
%	Copyright (c) 2015 Jan Küster
%
%	Permission is hereby granted, free of charge, to any person obtaining a copy
%	of this software and associated documentation files (the "Software"), to deal
%	in the Software without restriction, including without limitation the rights
%	to use, copy, modify, merge, publish, distribute, sublicense, and/or sell
%	copies of the Software, and to permit persons to whom the Software is
%	furnished to do so, subject to the following conditions:
%	
%	THE SOFTWARE IS PROVIDED "AS IS", WITHOUT WARRANTY OF ANY KIND, EXPRESS OR
%	IMPLIED, INCLUDING BUT NOT LIMITED TO THE WARRANTIES OF MERCHANTABILITY,
%	FITNESS FOR A PARTICULAR PURPOSE AND NONINFRINGEMENT. IN NO EVENT SHALL THE
%	AUTHORS OR COPYRIGHT HOLDERS BE LIABLE FOR ANY CLAIM, DAMAGES OR OTHER
%	LIABILITY, WHETHER IN AN ACTION OF CONTRACT, TORT OR OTHERWISE, ARISING FROM,
%	OUT OF OR IN CONNECTION WITH THE SOFTWARE OR THE USE OR OTHER DEALINGS IN
%	THE SOFTWARE.
%	
%
%-----------------------------------------------------------------------------------------------------------------------------------------------%


%============================================================================%
%
%	DOCUMENT DEFINITION
%
%============================================================================%

%we use article class because we want to fully customize the page and dont use a cv template
\documentclass[10pt,A4]{article}


%----------------------------------------------------------------------------------------
%	ENCODING
%----------------------------------------------------------------------------------------

%we use utf8 since we want to build from any machine
\usepackage[utf8]{inputenc}

%----------------------------------------------------------------------------------------
%	LOGIC
%----------------------------------------------------------------------------------------

% provides \isempty test
\usepackage{xifthen}

%----------------------------------------------------------------------------------------
%	FONT
%----------------------------------------------------------------------------------------

% some tex-live fonts - choose your own

% \usepackage[defaultsans]{droidsans}
% \usepackage[default]{comfortaa}
% \usepackage{cmbright}
% \usepackage[default]{raleway}
% \usepackage{fetamont}
\usepackage[default]{gillius}
% \usepackage[light,math]{iwona}
% \usepackage[thin]{roboto} 

% set font default
\renewcommand*\familydefault{\sfdefault}
\usepackage[T1]{fontenc}

% more font size definitions
\usepackage{moresize}

\usepackage{fontawesome}

%----------------------------------------------------------------------------------------
%	PAGE LAYOUT  DEFINITIONS
%----------------------------------------------------------------------------------------

%debug page outer frames
%\usepackage{showframe}			


%define page styles using geometry
\usepackage[a4paper]{geometry}

% for example, change the margins to 2 inches all round
\geometry{top=.8cm, bottom=-1.7cm, left=0.2cm, right=0.2cm}


%less space between header and content
\setlength{\headheight}{-5pt}


%customize entries left, center and right
%\lhead{}
%\chead{ \small{Jan Küster  $\cdot$ Consultant and Software Engineer $\cdot$  Bremen, Germany  $\cdot$  \textcolor{sectcol}{\textbf{info@jankuester.com}}  $\cdot$ +49 176 313 *** **}}
%\rhead{}


%indentation is zero
\setlength{\parindent}{0mm}

%----------------------------------------------------------------------------------------
%	TABLE /ARRAY DEFINITIONS
%---------------------------------------------------------------------------------------- 

%for layouting tables
\usepackage{multicol}
\usepackage{multirow}

%extended aligning of tabular cells
\usepackage{array}

\newcolumntype{x}[1]{%
        >{\raggedleft\hspace{0pt}}p{#1}}%


%----------------------------------------------------------------------------------------
%	GRAPHICS DEFINITIONS
%---------------------------------------------------------------------------------------- 

%for header image
\usepackage{graphicx}

%for floating figures
\usepackage{wrapfig}
\usepackage{float}
%\floatstyle{boxed} 
%\restylefloat{figure}

%for drawing graphics		
\usepackage{tikz}
\usetikzlibrary{shapes, backgrounds,mindmap, trees}


%----------------------------------------------------------------------------------------
%	Color DEFINITIONS
%---------------------------------------------------------------------------------------- 
\usepackage{transparent}
\usepackage{color}

%accent color
\definecolor{complcol}{RGB}{250,150,10}

%dark background color
\definecolor{bgcol}{RGB}{110,110,110}

%light background / accent color
\definecolor{softcol}{RGB}{225,225,225}

\definecolor{sectcol}{RGB}{0,120,150}


%Package for links, must be the last package used
\usepackage[hidelinks]{hyperref}
\usepackage{amstex}

%============================================================================%
%
%
%	DEFINITIONS
%
%
%============================================================================%

% returns minipage width minus two times \fboxsep
% to keep padding included in width calculations
\newcommand{\mpwidth}{\linewidth-\fboxsep-\fboxsep}


%----------------------------------------------------------------------------------------
% 	ARROW GRAPHICS in Tikz
%----------------------------------------------------------------------------------------

% a six pointed arrow poiting to the left
\newcommand{\tzlarrow}{(0,0) -- (0.2,0) -- (0.3,0.2) -- (0.2,0.4) -- (0,0.4) -- (0.1,0.2) -- cycle;}

% include the left arrow into a tikz picture
% param1: fill color
%
\newcommand{\larrow}[1]
{\begin{tikzpicture}[scale=0.58]
     \filldraw[fill=#1!100,draw=#1!100!black]  \tzlarrow
\end{tikzpicture}
}

% a six pointed arrow poiting to the right
\newcommand{\tzrarrow}{ (0,0.2) -- (0.1,0) -- (0.3,0) -- (0.2,0.2) -- (0.3,0.4) -- (0.1,0.4) -- cycle;}

% include the right arrow into a tikz picture
% param1: fill color
%
\newcommand{\rarrow}
{\begin{tikzpicture}[scale=0.7]
     \filldraw[fill=sectcol!100,draw=sectcol!100!black] \tzrarrow
\end{tikzpicture}
}

%----------------------------------------------------------------------------------------
%	custom sections
%----------------------------------------------------------------------------------------

% create a coloured box with arrow and title as cv section headline
% param 1: section title
%
\newcommand{\cvsection}[1]
{
    \colorbox{sectcol}{\mystrut \makebox[1\mpwidth][l]{
        \larrow{bgcol} \hspace{-8pt} \larrow{bgcol} \hspace{-8pt} \larrow{bgcol} \textbf{\textcolor{white}{\uppercase{#1}}}\hspace{4pt}
    }}\\
}

% create a coloured arrow with title as cv meta section section
% param 1: meta section title
%
\newenvironment{metasection}[1] {
    \vspace{6pt}
    \begin{center}
        \textcolor{white}{\large{\uppercase{#1}}}\\
        \normalsize
        \parbox{0.7\mpwidth}{\textcolor{white}    \hrule}}{
    \end{center}}

\newenvironment{metasection_title}[1] {
    \vspace{6pt}
    \begin{center}
        \linespread{2}{
    \selectfont\textcolor{white}{\fontsize{25}{25}{{#1}}}\\
        \normalsize
        \parbox{0.7\mpwidth}{\textcolor{white}}}}{
    \end{center}}

%----------------------------------------------------------------------------------------
%	 CV EVENT
%----------------------------------------------------------------------------------------

% creates a stretched box as cv entry headline followed by some paragraphs about 
% the work you did
% param 1:	event time i.e. 2014 or 2011-2014 etc.
% param 2:	event name (what did you do?)
% param 3:	institution (where did you work / study)
% param 4:	list of paragraphs outlining your contributions, see example usage below
%
\newcommand{\cvevent}[4]
{
    \vspace{10pt}
    \begin{tabular*}{1\mpwidth}{p{0.55\mpwidth}  x{0.42\mpwidth}}
        \textcolor{black}{\normalsize\textbf{#2}} & \textbf{\normalsize\textcolor{black}{#3}}, \textcolor{black}{#1}
    \end{tabular*}
    \vspace{0pt}
    \textcolor{softcol}{\hrule}
    \vspace{6pt}
    \cvlist {#4}
    \vspace{-6pt}
}

% formats a list of strings with variable length for use in `\cvevent`
% param 1: a list of strings outlining your contributions
\newcommand{\cvlist}[1] {
    \foreach \listitem in {#1}
        {
        \begin{tabular*}
        {1\mpwidth}{p{1\mpwidth}}
            \parbox{1\mpwidth}{\larrow{softcol} \listitem}
            \vspace{6pt}
        \end{tabular*}
    }
}

% creates a stretched box as 
\newcommand{\cveventmeta}[2]
{
    \mbox{\mystrut \hspace{87pt}\textit{#1}}\\
    #2
}

%----------------------------------------------------------------------------------------
% CUSTOM STRUT FOR EMPTY BOXES
%----------------------------------------- -----------------------------------------------
\newcommand{\mystrut}{\rule[-.3\baselineskip]{0pt}{\baselineskip}}


% use to vertically center content
% credits to: http://tex.stackexchange.com/questions/7219/how-to-vertically-center-two-images-next-to-each-other
\newcommand{\vcenteredinclude}[1]{\begingroup
\setbox0=\hbox{\includegraphics{#1}}%
\parbox{\wd0}{\box0}\endgroup}

% use to vertically center content
% credits to: http://tex.stackexchange.com/questions/7219/how-to-vertically-center-two-images-next-to-each-other
\newcommand*{\vcenteredhbox}[1]{\begingroup
\setbox0=\hbox{#1}\parbox{\wd0}{\box0}\endgroup}

%----------------------------------------------------------------------------------------
%	ICON-SET EMBEDDING
%---------------------------------------------------------------------------------------- 

% at this point we simplify our icon-embedding by simply referring to a set of png images.
% if you find a good way of including svg without conflicting with other packages you can
% replace this part
\newcommand{\icon}[3]{\makebox(#2, #2){\textcolor{#3}{\csname fa#1\endcsname}}}    %icon shortcut
\newcommand{\icontext}[4]{                        %icon with text shortcut
    \vcenteredhbox{\icon{#1}{#2}{#4}} \vcenteredhbox{\textcolor{#4}{#3}}
}
\newcommand{\iconhref}[5]{                        %icon with website url
    \vcenteredhbox{\icon{#1}{#2}{#5}} \href{#4}{\textcolor{#5}{#3}}
}

\newcommand{\iconemail}[5]{                        %icon with email link
    \vcenteredhbox{\icon{#1}{#2}{#5}} \href{mailto:#4}{\textcolor{#5}{#3}}
}


%============================================================================%
%
%
%
%	DOCUMENT CONTENT
%
%
%
%============================================================================%
\begin{document}

\fcolorbox{white}{sectcol}{\begin{minipage}[c] [0.95\textheight][t]{0.27\linewidth}
\vspace{30pt}
\begin{metasection_title}
{\bfseries Florian GUYOT}
\end{metasection_title}


\begin{metasection}{Contact}

    \icontext{MapMarker}{12}{92160 Antony}{white}\\[6pt]
    \icontext{MobilePhone}{12}{06 51 21 25 01}{white}\\[6pt]
    \iconemail{Send}{12}{axelflorian@free.fr}{axelflorian@free.fr}{white}\\[6pt]
    \iconhref{Github}{12}{https://github.com/florianguyot91}{https://github.com/florianguyot91}{white}\\[6pt]

\end{metasection}

%----------------------------------------------------------------------------------------
%	META SECTION
%----------------------------------------------------------------------------------------

\begin{metasection}{Compétences}

    \icontext{}{-12}{Full Stack Developpement}{white}\\[6pt]
    \icontext{}{-12}{Réseaux de neuronnes}{white}\\[6pt]
    \icontext{}{-12}{Cloud Computing}{white}\\[6pt]
    \icontext{}{-12}{Informatique quantique}{white} \\[6pt]


\end{metasection}


\begin{metasection}{Langues}

    \textcolor{white}{
        \icontext{}{-12}{Anglais C1 (TOEIC 945)}{white} \\[6pt]
        \icontext{}{-12}{Espagnol B1}{white} \\[6pt]
        \icontext{}{-12}{Japonais A2}{white} \\[6pt]
    }
\end{metasection}


\begin{metasection}{Technologies}

    \textcolor{white}{
        \icontext{}{-12}{Docker}{white} \\[6pt]
        \icontext{}{-12}{UNIX/Linux}{white} \\[6pt]
        \icontext{}{-12}{Git}{white} \\[6pt]
    }
\end{metasection}

\begin{metasection}{Logiciels}

    \textcolor{white}{
        \icontext{}{-12}{JetBrains IDE's}{white}\\[6pt]
        \icontext{}{-12}{VSCode}{white}\\[6pt]
        \icontext{}{-12}{MATLAB}{white}\\[6pt]
        \icontext{}{-12}{Unity}{white}\\[6pt]
    }
\end{metasection}

\begin{metasection}{Activités}

    \textcolor{white}{
    \icontext{}{-12}{Conception 3D (Blender)}{white}\\[6pt]
    \icontext{}{-12}{Montage vidéo (DaVinci)}{white}\\[6pt]
    \icontext{}{-12}{Batterie (groupe de métal)}{white}\\[6pt]
    \icontext{}{-12}{Guitare}{white}\\[6pt]
    \icontext{}{-12}{Piano}{white}\\[6pt]
    \icontext{}{-12}{Plongée}{white}\\[6pt]
    }
\end{metasection}


%---------------------------------------------------------------------------------------
%	QR CODE (optional)
%----------------------------------------------------------------------------------------

\vspace{12pt}
%\begin{center}
%\includegraphics[width=0.35\mpwidth]{qrcode}
%\end{center}

                                 \end{minipage}}
\fcolorbox{white}{white}{\begin{minipage}[c][0.95\textheight][t]{0.69\linewidth}


%---------------------------------------------------------------------------------------
%	TITLE HEADLINE
%----------------------------------------------------------------------------------------
% use this for multiple words like working titles etc.
%\hspace{-0.25\linewidth}\colorbox{bgcol}{\makebox[1.5\linewidth][c]{\hspace{46pt}\HUGE{\textcolor{white}{\uppercase{M.Sc. Jan Küster}} } \textcolor{sectcol}{\rule[-1mm]{1mm}{0.9cm}} \parbox[b]{5cm}{   \large{ \textcolor{white}{{IT Consultant}}}\\
% \large{ \textcolor{white}{{JS Fullstack Engineer}}}}
%}}

% use this for single words, e.g. CV or RESUME etc.
%----------------------------------------------------------------------------------------
%	HEADER IMAGE
%----------------------------------------------------------------------------------------


%\hspace{-1.6cm}
%\includegraphics[trim= 0 250 0 270,clip,width=1\linewidth+3.1cm]{myfoto.jpg}	%trimming relative to image size!
%\includegraphics[trim= 350 150 0 200, clip ,width=\linewidth]{myfoto.jpg}	%trimming relative to image size

%---------------------------------------------------------------------------------------
%	SUMMARY
%----------------------------------------------------------------------------------------
%\transparent{0.85}%
%\vspace{70pt}
%\hspace{0.4\linewidth}
%\colorbox{bgcol}{
%	\parbox{0.5\linewidth}{
%		\transparent{1}%
%		\begin{center}
%		\larrow{sectcol}\larrow{sectcol}\textcolor{white}{Mon taff est de trouver un p****n de master avant fin mai.}
%		\end{center}
%	}
%}
%============================================================================%
%
%	CV SECTIONS AND EVENTS (MAIN CONTENT)
%
%============================================================================%

%---------------------------------------------------------------------------------------
%	STATUS
%----------------------------------------------------------------------------------------
\vspace{20pt}
\fontsize{15}{15}
\centering{\bfseries ÉTUDIANT EN 5 ÈME ANNÉE A L'EPF


MAJEURE INGÉNIERIE ET NUMÉRIQUE
}

                                 \vspace{25pt}

%---------------------------------------------------------------------------------------
%	EXPERIENCE
%----------------------------------------------------------------------------------------
                                 \cvsection{Expérience professionnelle}

%
                                 \cvevent{09/2022-01/2023}
                                 {Stage élève ingénieur}
                                 {Fraunhofer ISI}
                                 { Développement du back end d'une application web mandatée par la Commission Européenne \newline\hspace*{4mm}en collaboration avec différents instituts (MICAT) :
                                 \newline
                                 \hspace*{4mm}- Pandas pour la gestion de dataframe et de lecture de JSON
                                 \newline
                                 \hspace*{4mm}- Numpy pour les opérations économiques
                                 \newline
                                 \hspace*{4mm}- Flask pour la communication entre le back et le front end
                                 \newline
                                 \hspace*{4mm}- Collaboration avec Git
                                 \newline
                                 \iconhref{}{12}{https://micatool.eu/micat-project-en/index.php}{https://micatool.eu/micat-project-en/index.php}{black}\\[6pt]
                                 }

%\textcolor{softcol}{\hrule}


%\cvevent{2014 - 2016}{IT Consultant for IBM XPages and Notes Domino}{We4IT GmbH Bremen}{Realize projects in XPages and We4IT Aveedo, monitor project status, conduct reports}{Integrated Camunda BPMN engine and BPMN.IO modeler in We4IT Aveedo}


%\textcolor{softcol}{\hrule}


                                 \cvevent{08/2021}
                                 {Stage engagement citoyen}
                                 {Zen 2050 Maintenant}
                                 {Conception sous WordPress et mise en ligne du site web de l’association
                                 \newline
                                 \iconhref{}{12}{https://zen2050maintenant.net/}{https://zen2050maintenant.net/}{black}\\[6pt]
                                 }

%\textcolor{softcol}{\hrule}

%
                                 \cvevent{03/2021-07/2021}
                                 {CDD Assistant technicien}
                                 {Praxea-Diagnostics}
                                 {Découpe et préparation d'échantillons organiques pour le diagnostic médical}

%\textcolor{softcol}{\hrule}


%
                                 \cvevent{07/2019}
                                 {Stage ouvrier}
                                 {GEORGIN}
                                 {Montage de composants électroniques sur des cartes{,} (soudures{,} assemblage{,} moulage), Contrôle qualité du produit final}

                                    \vspace{25pt}
%---------------------------------------------------------------------------------------
%	EDUCATION SECTION
%--------------------------------------------------------------------------------------
                                 \cvsection{Projets - Diplômes}

                                 \cvevent{2018-2024}
                                 {Ecole d’ingénieur généraliste}
                                 {EPF de Cachan}
                                 {Site web utilisant Java Springboot pour le back{,} Angular pour le front et Docker pour la BDD,
                                     Construction d'un réseau de neurones pour trier des images,
                                     Projet chez Deloitte visant à intégrer un module de bases de données dans SAP,
                                     Développement d'une application mobile en Kotlin depuis l'API The Movie Database,
                                     Conception d'une IA en Python (Tensorflow keras) pour génerer des maps dans un jeu de \hspace*{4mm}rythme (Osu!),
                                     Création d'un site en JAVA de gestion d'une flotte de voitures associée à une base clients \hspace*{4mm} et réservations,
                                     Développement d'un site web en JS/PHP de prise de rendez-vous et gestion clients d'un \hspace*{4mm} cabinet médical,
                                     Création d’une application web d'un jeu de mémoire (front end{,} back end),
                                     Gestion de bases de données en SQL avec UML,
                                     Design de circuits quantiques (Addition{,} ect),
                                     Approximations d'intégrales sous MATLAB,
                                     Modélisation et simulation de contraintes d’un vérin hydraulique en 3D avec CATIA V5,
                                     Analyse comparée des flux thermiques dans une ailette de refroidissement{,} entre un modèle \hspace*{4mm}physique et une simulation sous MATLAB/COMSOL,
                                     Création d'un prototype de borne de stationnement de voitures en ARDUINO et gestion \hspace*{4mm} d'autorisation depuis une page web,
                                     Programmation d’un Raspberry pour commander une voiture et la contrôler à distance
                                 }

                                 \cvevent{2018}
                                 {Lycée Saint-Charles}
                                 {Athis-Mons}
                                 {Bac S option Informatique et Science du Numérique,
                                     Développement d'un jeu d'arène avec "path finding"}



% \textcolor{softcol}{\hrule}

%
%\cvevent{2012 - 2013}{Master Project - PrIMA}{University of Bremen}{Co-Invented a touch table application for medical support{,} co-developed software (Java), Formed a scrum team{,} mainted project dev server (Debian){,} surveyed target audience}
%
%%\textcolor{softcol}{\hrule}
%
%%
%\cvevent{2012 - 2015}{Master Studies Digital Media}{University of Bremen}{Inter-cultural classes in English{,} covering special topics in computer science and design, Professionalized in research methods{,} software development and e-assessment}
%
%%\textcolor{softcol}{\hrule}
%
%%
%\cvevent{2009 - 2010}{Semester Abroad}{University of Melbourne}{Mastered six months of study and trans-cultural experience in Melbourne{,} Australia, Finished machine programming{,} information visualization{,} professional essay writing}


    \end{minipage}}%

%-------------------------------------------------------------------------------------------------
%	ARTIFICIAL FOOTER (fancy footer cannot exceed linewidth) 
%--------------------------------------------------------------------------------------------------

%\null
%\vspace*{\fill}
%\hspace{-0.25\linewidth}\colorbox{bgcol}{\makebox[1.5\linewidth][c]{\mystrut \small \textcolor{white}{Coypright 2018 jkuester@uni-bremen.de} $\cdot$ \textcolor{white}{licensed under MIT license}}}\\[-6pt]

%============================================================================%
%
%
%
%	DOCUMENT END
%
%
%
%============================================================================%
\end{document}
