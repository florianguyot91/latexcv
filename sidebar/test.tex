\documentclass[12pt]{lettre}

\usepackage[utf8]{inputenc}
\usepackage[T1]{fontenc}
\usepackage{lmodern}
\usepackage{eurosym}
\usepackage[frenchb]{babel}
\usepackage{numprint}

\name{Paul \textsc{Bismuth}}
\signature{Paul \textsc{Bismuth}}
\address{Paul \textsc{Bismuth}\\30 rue de Paris\\80886 Sassonne-le-Creux}
\lieu{Sassonne-le-Creux}
\telephone{06 07 08 09 10}
\email{paul.bismuth@sassonne.fr}
\nofax

\begin{document}

    \begin{letter}{Un destinataire\\Une addresse\\12345 Une ville}

        \def\concname{Objet :~} % On définit ici la commande 'objet'
        \conc{ceci est mon objet}
        \opening{DUDE,}

        Étudiant en quatrième année à l’EPF de Cachan, en majeure Ingénierie et Numérique et très intéressé
        par le développement de projets en lien avec la physique quantique,
        je candidate à votre master « Quantum and Distributed Computer Science ».

        Ma formation en informatique m'a permis d'acquérir une base de connaissances en programmation,
        en algorithmique et en mathématiques.
        J'ai également suivi des cours en informatique quantique, me permettant de comprendre et de développer des compétences
        sur des plateformes telles qu'IBM Quantum et la bibliothèque Qiskit.
        Ces connaissances préliminaires me motivent à approfondir mes acquis dans ce domaine
        en rejoignant votre programme de master.

        En effet, j'aspire à concevoir et gérer des super-calculateurs basés sur des processeurs quantiques.

        Je vous remercie sincèrement de l'attention portée à ma candidature et
        je me tiens à votre disposition pour toute information complémentaire dans la perspective
        d'intégrer votre master de septembre à janvier 2024.

        \closing{Veuillez agréer, DUDE, mes salutations distinguées,}

    \end{letter}

\end{document}